\section{Goal}
\label{sec:Goal}

The goal of this analysis is the classification of $B_s$ candidates in a data samples from the LHCb experiment. This is achieved by employing simulation samples next to the data samples to train a  multivariate classifier.  

\section{Theory}
\label{sec:Theory}

Before analyzing the data sample a look at the studied decay \signal it is important to understand it's kinematic. The $B_s$ usually decays weakly as seen in \autoref{fig:feynman}

\begin{figure}[htpb]
	\centering
	\def\widthscale {0.08\textwidth}
	\def\heightscale {0.04\textwidth}
	\usetikzlibrary{shapes.misc}
	\tikzset{cross/.style={cross out,  draw=black, minimum size=2*(####1-\pgflinewidth), inner sep=0pt, outer sep=0pt},
		%default radius will be 4pt. 
		cross/.default={5pt}}
	\begin{tikzpicture}
		\begin{feynman}[small]
			\vertex(a1) {\(\bar{b}\)};
			\vertex[right=3.15*\widthscale of a1] (a2){\(\bar{c}\)} ;
			
			\vertex[below=3*\heightscale of a1] (l2) {\(s\)};
			\vertex[right=1.5*\widthscale of a1] (rm2);
			\vertex[below=1*\heightscale of a2] (r2) {\(c\)};
			\vertex[right=0.45*\widthscale of rm2](tmp1);
			\vertex[below=1.5*\heightscale of tmp1] (rm3);
			\vertex[below=1*\heightscale of r2] (r3) {\(\bar{d}\)};
			\vertex[below=1*\heightscale of r3] (r4) {\(s\)};
			
			
			
			\diagram* {
				(a1) -- [fermion, ] (rm2),
				(rm2) -- [fermion, ] (a2),
				(rm2)-- [photon, bend right,edge label'={\(W^+\)}] (rm3),
				(rm3) -- [fermion] (r2),
				(r3) -- [fermion] (rm3),
				(l2) -- [fermion] (r4),
			};
			\draw [decoration={brace}, decorate, thick] (l2.south west) -- (a1.north west) node [pos=0.5, left] {$B_s\ $};
			\draw [decoration={brace}, decorate, thick] (a2.north east) -- (r2.south east) node [pos=0.525, right] {$\ \Psi(2S)$};
			\draw [decoration={brace}, decorate, thick] (r3.north east) -- (r4.south east) node [pos=0.5, right] {$\ K_s$};            
		\end{feynman}
	\end{tikzpicture}
	\label{fig:feynman}
	\caption{Feynmangraph of the decay \signal.}
\end{figure}



