\section{Discussion}
\label{sec:Discussion}

Some remarks on the analysis should be made to put the final result into more context.
While the distributions of the chosen variables show good seperation, there is still room for optimization. The number of variables was arbitrarily chosen as 16,
but can easily be varied when changing the requirements on the Kolmogorov Smirnov test statistics or correlations. Testing the BDT performance with different number of
parameters could help to improve its predictive power. The five individual BDTs performed very similar, indicating that there is not too much overtraining.
However, the BDTs have all been initialized with default parameters. Tuning these parameters in a grid search could definetely help improve the BDT performance and thus
the final result.
The fits on the signal peaks have been modeled by two gaussian functions, however they cannot describe the assymetric shape of these peaks perfectly.
A more complicated model like a double-sided crystall ball function can improve the fit and thus directly the yield of the signal.
The significance was only estimated using \eqref{eq:sig} and seems rather high beeing greater than $5 \, \sigma$. Including uncertainties should help to mitigate this fact.