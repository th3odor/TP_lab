\newpage
\section{Analysis}
\label{sec:Analysis}

\subsection{Spectrometer Measurement}

Initially, the two measurements of the spectra are plotted and shown in \autoref{fig:lightonoff}. 
These show the effect the light has on a measurement.

\begin{figure}[H]
	\centering
	\includegraphics[width=0.7\linewidth]{build/LightOnOff}
	\caption{Original recorded wavelength spectra and their respective counts at a fixed distance and fixed angles.}
	\label{fig:lightonoff}
\end{figure}

To reduce the effect of external light sources, the counts during a measurement with the activated laser are subtracted by the
so-called dark counts (measurements with the laser deactivated). The newly cleaned spectra are shown in \autoref{fig:dcsubtracted}.

\begin{figure}[H]
	\centering
	\includegraphics[width=0.7\linewidth]{build/DCsubtracted}
	\caption{Wavelength spectra and their respective counts with the dark counts removed at a fixed distance and fixed angles.}
	\label{fig:dcsubtracted}
\end{figure}

\subsection{Radial Symmetry measurement}
The goal of this part of the analysis is the visualization of the light intensity in relation to the horizontal and vertical measurement angles.
The intensity is determined as the sum of the counts of a measured spectrum with the dark counts subtracted. 
Additionally, each intensity is normed by dividing each by the highest measured intensity and plotted in a two-dimensional histogram shown in \autoref{fig:angles}

\begin{figure}
	\centering
	\includegraphics[width=0.7\linewidth]{build/angles}
	\caption{Angular distribution of the measured intensity in relation the horizontal and vertical angle.}
	\label{fig:angles}
\end{figure}

\subsection{Simulation}

The simulated data is created using the software \texttt{GEANT4} and consists of 1200 Files with 14 features each. 
These are used to gain insight into the inner workings of the fiber. 
\subsubsection{Simulation cleaning}
Due to errors in the simulation process, some of the generated events have to be removed. 
This includes photons arriving at the photomultiplier outside of the fiber as well as photons that caused Rayleigh scattering. 
The effect of the wrong generated measurements and the application of a requirement on the calculated \texttt{r\_exit} variable 
(distance to the core of the fiber) $\mathtt{r\_exit}\leq \qty{0.125}{\milli\meter}$ is shown in \autoref{fig:rcut}.

\begin{figure}
	\centering
	\includegraphics[width=0.7\linewidth]{build/rCut}
	\caption{Distribution of the exit distances $r_\mathrm{min}$ in the simulation sample before and after the applied requirement (red line).}
	\label{fig:rcut}
\end{figure}

Additionally, the simulation data is split into two samples for core and cladding photons. Where measurements of core photons have a \texttt{length\_clad} = 0 and cladding photons \texttt{length\_clad} $\neq$ 0.

\subsubsection{Production Angle}
Before plotting the angles of the theoretical maximum angles are determined using the formulas

\begin{align*}
	\theta_1 &= \arccos\left(\frac{n_2}{n_1}\right) = \qty{21.37}{\degree},\ \mathrm{and}\\
	\theta_2 &= \arccos\left(\frac{n_3}{n_1}\right) = \qty{27.44}{\degree},
\end{align*}
where $n_1$, $n_2$, and $n_3$ are the refraction indices inside the scintillating fiber.
Following the aforementioned step, the angles $\theta$ of the simulation data have to be determined by 
calculating the angle between the momentum of the generated photon and the $x$-axis. 
The histogrammed angles of both simulation data samples are shown in \autoref{fig:anglesmantlecore} along the two theoretically determined angles.
%LINEAR FRAGE BEANTWORTEN


\begin{figure}[H]
	\centering
	\includegraphics[width=0.65\linewidth]{build/AnglesMantleCore}
	\caption{The angular distribution for both the core and cladding photons and the calculated maximum angles $\theta_1$ and $\theta_2$.}
	\label{fig:anglesmantlecore}
\end{figure}


\subsubsection{Minimum distance}

For the next step, the minimum distance between the path of the photon and the $x$-axis is calculated using the formula 
\begin{equation*}
	r_\mathrm{min} = \left|\frac{z*p_y - p_zy}{\sqrt{p_z^2 + p_y^2}}\right|,
\end{equation*}
where $y$, $z$, $p_y$, and $p_z$ are the positions and momenta of the photon at its creation. 
This formula is obtained by calculating the minimal distance of the photons path (starting point $x$, $y$, $z$ and direction $p_x$, $p_y$, $p_z$) from the$x$-axis. 
Using the newly created variable and the already determined angle a two-dimensional histogram is created to visualize their relation. 
The histogram for the core photons is shown in \autoref{fig:rminCore} and for the cladding photons in \autoref{fig:rminMantle}.
%REL ERKLEREN

\begin{figure}[H]
	\centering
	\begin{subfigure}{0.45\textwidth}
		\includegraphics[width=\textwidth]{build/rminCore.pdf}
		\caption{Core photons}
		\label{fig:rminCore}
	\end{subfigure}
	\begin{subfigure}{0.45\textwidth}
		\includegraphics[width=\textwidth]{build/rminMantle.pdf}
		\caption{Cladding photons}
		\label{fig:rminMantle}
	\end{subfigure} 
	\caption{Two-dimensional histograms of the minimal distance $r_\mathrm{min}$ and angle $\theta$.}
	\label{fig:rmin}
\end{figure}
\newpage
\subsubsection{Intensity}
The last study performed on the simulation samples focuses on the intensity relation described in \eqref{eq:Attentuation}. 
The aforementioned relation is visualized in \autoref{fig:int2d} using a two-dimensional histogram of the angle 
$\theta$ and the \texttt{gpsPosX} variable ($x$ position of the excitation in the fiber). 


\begin{figure}[H]
	\centering
	\includegraphics[width=0.7\linewidth]{build/Int2d}
	\caption{Two-dimensional histogram of the intensity distribution in relation of the angle $\theta$ and the $x$ position of the excitation in the fiber.}
	\label{fig:int2d}
\end{figure}

Using the counts determined from the histogram multiple fits of the intensity in relation to the \texttt{gpsPosX} variable are performed. %For fitting the python package \cite{scif}
The function 
\begin{equation}
	I(x) = I_0 \exp\left(-ax\right)
\end{equation}
used for the fits, results in the plots shown in \autoref{fig:simfits}.

\begin{figure}[H]
	\centering
	\includegraphics[width=0.7\linewidth]{build/simfits}
	\caption{Fits on the determined intensities of the simulation samples against the \texttt{gpsPosX} variable at different angles $\theta$.}
	\label{fig:simfits}
\end{figure}

The determined fit parameters $a$ and their corresponding angles $\theta$ are then plotted in \autoref{fig:simthetafit} alongside the theoretical curve for $a_{\mathrm{eff}}$ from \eqref{eq:Attentuation}.

\begin{figure}
	\centering
	\includegraphics[width=0.7\linewidth]{build/simThetaFit}
	\caption{The determined fit parameters $a$ with their respective angle and the expected curve.}
	\label{fig:simthetafit}
\end{figure}

\subsection{Intensity Measurement}
Similarly to the intensity fits for the simulation the measured intensity is fitted against the \texttt{gpsPosX} variable. % and the determined fit parameters used to determine
The determined fits are shown in \autoref{fig:datafits}. Since the measurements don't follow the expected exponential slope, two fits per angle are performed and the mean of the fit parameter is used for further analysis.


\begin{figure}
	\centering
	\includegraphics[width=0.7\linewidth]{build/dataFits}
	\caption{Fits on the measured intensities against the \texttt{gpsPosX} variable at different angles $\theta$.}
	\label{fig:datafits}
\end{figure}

The determined parameters $a$ are then again plotted against the angle $\theta$. On these parameters another function, of the form 
\begin{equation*}
	a(\theta) = \frac{A}{\cos(\theta)} + B \tan(\theta),
\end{equation*}
is fitted. The parameters and the corresponding fit are shown in \autoref{fig:datathetafit}

\begin{figure}
	\centering
	\includegraphics[width=0.7\linewidth]{build/DataThetaFit}
	\caption{Fit on the determined fit parameters $a$ against their respective angles $\theta$ and the outlier (red).}
	\label{fig:datathetafit}
\end{figure}
The data point marked red in \autoref{fig:datathetafit} lies far from the expected curve, so another fit without the outlier is performed in \autoref{fig:datathetafitalt}.

\begin{figure}[H]
	\centering
	\includegraphics[width=0.7\linewidth]{build/DataThetaFitAlt}
	\caption{Fit on the determined fit parameters $a$ against their respective angles $\theta$ with the outlier excluded.}
	\label{fig:datathetafitalt}
\end{figure}

The performed fits result in the parameters shown in \autoref{tab:fitparam}.

\begin{center}
	\begin{tabular}{c c c}
		\toprule
		Parameter & Fit & Fit - Outlier excluded \\
		\midrule
		$A$ & \qty{1.548\pm0.435e-04}{\meter\tothe{-1}} & \qty{1.352\pm0.130e-04}{\meter\tothe{-1}} \\
		$B$ & \qty{-4.850\pm11.471e-05}{\meter\tothe{-1}} & \qty{8.323\pm1.30e-05 }{\meter\tothe{-1}} \\
		\bottomrule
		\label{tab:fitparam}
	\end{tabular}
\end{center}

\subsection{Angle Intensity Measurement}
For the final part of the analysis, the measured angle with the highest intensity is determined. 
The intensities for each angle are again calculated as the sum of the cleaned counts of each measurement. 
The angle determined to have the highest intensity is $\theta = \qty{14.5}{\degree}$. 
Both the angular distribution and the maximum angle are shown in \autoref{fig:maxangle}.

\begin{figure}
	\centering
	\includegraphics[width=0.7\linewidth]{build/MaxAngle}
	\caption{Intensity distribution of the measured angle and the determined maximum angle.}
	\label{fig:maxangle}
\end{figure}
