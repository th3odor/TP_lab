\section{Analysis}
\label{sec:Analysis}

\subsection{Spectrometer Measurement}

Initially the two measurements of the spectra are plotted and shown in \autoref{fig:lightonoff}. 
These show the effect the light has on a measurement.

\begin{figure}[H]
	\centering
	\includegraphics[width=0.7\linewidth]{build/LightOnOff}
	\caption{}
	\label{fig:lightonoff}
\end{figure}

To reduce the effect of external light sources, the counts during a measurement with the activated laser are subtracted by the
so-called dark counts (measurements with the laser deactivated). The new cleaned spectra are shown in \autoref{fig:dcsubtracted}.

\begin{figure}[H]
	\centering
	\includegraphics[width=0.7\linewidth]{build/DCsubtracted}
	\caption{}
	\label{fig:dcsubtracted}
\end{figure}

\subsection{Radial Symmetry measurement}
The goal of this part of the analysis is the visualization of the light intensity in relation to the horizontal and vertical measurement angle.
The intensity is determined as the sum of the counts of a measured spectrum with the dark counts subtracted. 
Additionally each intensity is normed by dividing each by the highest measured intensity and plotted in a two dimensional histogram shown in \autoref{fig:angles}

\begin{figure}
	\centering
	\includegraphics[width=0.7\linewidth]{build/angles}
	\caption{}
	\label{fig:angles}
\end{figure}

\subsection{Simulation}

The simulated data is created using the software \texttt{GEANT4} and consist of 1200 Files with 14 features each. These are used to gain a insight into the inner workings of the fiber. 
\subsubsection{Simulation cleaning}
Due to errors in the simulation process some of the generated events have to be removed. 
This includes photons arriving at the photomultiplier outside of the fiber as well as photons that caused Rayleigh scattering. 
The effect of the wrong generated measurements and the application of a requirement on the calculated \texttt{r\_exit} variable 
(distance to the core of the fiber) $\mathtt{r\_exit}\leq \qty{0.125}{\milli\meter}$ is shown in \autoref{fig:rcut}.

\begin{figure}
	\centering
	\includegraphics[width=0.7\linewidth]{build/rCut}
	\caption{}
	\label{fig:rcut}
\end{figure}

Additionally the simulation data is split into two samples for core and cladding photons. Where measurements of core photons have a \texttt{length\_clad} = 0 and cladding photons \texttt{length\_clad} $\neq$ 0








































