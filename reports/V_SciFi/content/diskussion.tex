\section{Discussion}
\label{sec:Discussion}

The plots of the spectrum measurement show the effect of the roomlight very well. Also, the subtraction of the dark counts result in two very similar distributions. Only the measurement performed with the roomlight turned on shows some noise underlying the importance of a dark environment for the experiment.\\
When looking at \autoref{fig:angles}, the maximum intensity is expected at \qty{0}{\degree} in both axes. In this case, the maximum is shifted by a few degrees, which may be attributed to a slightly misaligned fiber or measuring apparatus.\\
The maximum angles in the angular distribution shown in \autoref{fig:anglesmantlecore} are in agreement with the calculated ones, showing only a minimal shift. 
The intensity also seems to increase linearly with the angle until total reflection is reached. This is expected, as the reflected intensity is proportional to $\sin(\theta)$ as the photons are created equally distributed in the solid angle. And for small angles the $\sin$ appears to increase linear.\\%will be lower if the production angle deviates from the total reflection angle.\\
Especially clean results can be observed in \autoref{fig:rmin}. The distributions show that the cladding photons only have a distinct range in relation to their minimal distance and are cut off at an angle of about \qty{21}{\degree}. On the other hand, the core photons take over where no more cladding photons can be found, and allow for angles lower than the cladding ones. Additionally, \autoref{fig:rminMantle} shows the fact that a higher angle requires a higher $r_\mathrm{min}$ value for the photon to still perform total reflection. This behavior is in good agreement with the expected result.\\
The first inconsistencies in the simulation samples can be observed during the determination of the angular dependence of the fit parameter $a$. For the simulation samples, the fit parameter shows no dependence on the angle. The explanation for this needs further information on the simulation method applied in the \texttt{GEANT4} software. Due to this, no function was fitted on the obtained result in order to obtain a value on the minimal distance.\\
A similar fit was performed on the measured samples leading to more meaningful results. Only one of the measurements resulted in an outlier. 
This outlier can be easily explained by a sudden jump in the intensity for $\theta=\qty{32}{\degree}$ in \autoref{fig:datafits}. 
The origin of the jump may have resulted from a disturbance of the experimental setup during the measurement. 
Additionally, the drop in intensity shown in \autoref{fig:datafits} results from kink in the fiber at around \qty{50}{\centi\meter}.\\
Lastly, the angular intensity measurement in \autoref{fig:maxangle} shows no clear maximum. The maximum angle was determined as \qty{29.5}{\degree}, although this could also be a statistical fluctuation.
