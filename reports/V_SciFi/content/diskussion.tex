\section{Discussion}
\label{sec:Discussion}

The plot of the spectrum measurement show the effect of the light very well. Also, the subtraction of the dark counts result in two very similar distributions. Only the measurement performed with the light turned on shows some noise underlying the importance of a dark environment for the experiment.\\
When looking at \autoref{fig:angles}, the maximum intensity is expected to lay at \qty{0}{\degree} in both axes. In this case, the maximum is shifted by a few degrees, which may be attributed to a slightly misaligned fiber or measuring apparatus.\\
The maximum angles in the angular distribution, shown in \autoref{fig:anglesmantlecore} are in agreement with the calculated ones, with only a minimal shift. 
The intensity also seems to increase linearly with the angle. This is expected, as the reflected intensity will be lower the further the angle deviates the total reflection angle.\\ %LINEARER ANSTIEG
Especially clean results can be observed in \autoref{fig:rmin}. The distributions show that the cladding photons only have a distinct range in relation to their minimal distance and are cut off at an angle of about \qty{21}{\degree}. On the other hand, the core photons take over where no more cladding photons can be found, and allow for angles lower than the cladding ones. This behavior is in good agreement with the expected result.\\
The first inconsistencies in the simulation samples can be observed during the determining of the angular dependence of the fit parameter $a$. For the simulation samples, the fit parameter shows no dependence on the angle. The explanation for this needs further information on the simulation method applied in the \texttt{GEANT4} software. Due to this no function was fitted on the obtained result in order to obtain a value on the minimal distance.\\
A similar fit was performed on the measured samples leading to more meaningful results. Only one of the measurement series resulted in an outlier. 
This outlier can be easily explained by a sudden jump in the intensity for $\theta=\qty{32}{\degree}$ in \autoref{fig:datafits}. 
The origin of the jump though may have resulted from a disturbance of the experimental setup during the measurement. 
Additionally drop in the intensity shown in \autoref{fig:datafits} results from kink in the fiber at around \qty{50}{\centi\meter}.\\
Lastly, the angular intensity measurement in \autoref{fig:maxangle} shows no clear maximum, so the determined maximum angle of \qty{14.5}{\degree} was chosen as having the highest intensity.\\
All in all, the performed experiment led to a range of different results. Especially unfortunate is the abundance of an angular dependency in \autoref{fig:simthetafit}.
