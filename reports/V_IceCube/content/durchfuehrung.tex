\section{The IceCube Experiment}
\label{sec:The_IceCube_Experiment}
The IceCube experiment is located in the antarctic and consists of a huge array of photomultiplier inside the ice. 
It's goal is the measurement of origin and energy of astrophysical neutrinos.
The experiment is subdivided into the in-ice array, DeepCore, and IceTop. 
This subdivision is due to the different densities of photomultiplier on a cable. 
The DeepCore for example has the highest photomultiplier density and allows for measurements in a lower energy 
range (\qty{10}{\giga\eV}), where as the in-ice array limit lays around \qty{100}{\giga\eV}.
Additionally the IceTop, positioned above the DeepCore and the in-ice array, 
is used as a shower detector and simultaneously as a veto for Conventional neutrinos.
In total 5160 photomultiplier are placed at a depth of 1450 - 2450\,\unit{\meter}. 
These are used in order to measure the Cherenkov light emitted by charged particles exceeding the speed of light inside the medium it is traversing. \\

Since neutrinos only interact via the weak interaction a direct measurement is basically imposable. 
Therefore the IceCube experiment looks for particles created by weak interactions of the neutrinos. 
In this case through charged and neutral currents (CC and NC) as seen in \autoref{eq:CC} (CC) and \autoref{eq:NC} (NC).
\begin{align}
	\barparen{\nu_l} + A &= l^\mp + X \label{eq:CC}\\
	\nu_l + A &= \nu_l + X \label{eq:NC}	
\end{align}
Leptons created in CC are detected using the shape of the signature they leave in the detector. 
Electronic signals appear round whereas the signal generated by neutrinos 
are longer as they loose tier energy slower than an electron. 
Tau signatures are harder to detect as they generate two round signals, 
one from the fast decaying tau, the second form the electron generated from the tau decay.
Particles from NC, are detected by cascades created from hadronic particles, 
which leave a signature similar to the electronic one.\\

The different types of signatures also impact the quality of the measurement. Muons which created outside of the detector but still leave a track have a high energy uncertainty. The measurement of the angle though is much more precise compared to the electron. On the other hand, the electrons energy measurement carries a much smaller uncertainty.\\

The data samples used in this analysis are generated in Monte Carlo Simulations of the experiment. This allows for labeling of the so-called pseudo data. This data is used in the training of the applied algorithms.

\section{Tasks}
\label{sec:Tasks}

