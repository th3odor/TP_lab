\section{Data analysis}
\label{sec:Auswertung}

In total, three different data sets are available for this analysis. 
The training of the classifiers will be performed using the two datasets
\texttt{signal\_train.csv} and \texttt{background\_train.csv}, while the
\texttt{test.csv} file will be used to test the created model.

\subsection{Data preprocessing}

The events in the signal and background training datasets were produced via Monte Carlo simulations.
Consequently, the datasets contain Monte Carlo truths, event ids and weights which should not be
used during the training process, as they are not present in actual data.
All attributes with a name containing the key words \texttt{Weight}, \texttt{MC}, \texttt{Corsika}, or
\texttt{I3EventHeader} are removed from the datasets.
The training data is labeled with $0$ for background and $1$ for signal, these labels are also removed and 
stored, as they should only be used for validation purposes. Some entries contain \texttt{Nans} or
\texttt{Infs}, which have to be removed. As a first step, all attributes where more than \qty{10}{\percent}
of events are \texttt{Nans} or \texttt{Infs} are removed. The remaining invalid values are adressed by
deleting all events where any attribute is \texttt{Nan} or \texttt{Inf}.
Lastly, the data is rescaled by removing the mean and scaling to unit variance.

\subsection{Attribute selection}

Before the multivariate section can be performed, a selection of attributes is needed wich the classifier
will be using. The goal is to choose attributes which contain the most information about wether an event
is signal or background. Using all available attributes is avoided as this can lead to overtraining and
increases computational time.
The selection is performed using the mRMR method described in \autoref{sec:mRmR}. Using the package
\texttt{mrmr\_selection} \cite{mrmr}, the $100$ best features are determined in ascending order.
In \autoref{fig:features}, the distributions of the four best features for signal and background events
are shown.

\begin{figure}[H]
  \centering
  \includegraphics[width=1\textwidth]{build/features.pdf}
  \caption{Distributions of the four best features determined via the mRMR method, for signal and background events.}
  \label{fig:features}
\end{figure}

\subsection{Multivariate selection}

In this section, three different models are trained and compared. 

\subsubsection{Naives Bayes}

The first model used is the naives bayes classifier. To determine the optimal amount of
features, the model is trained and tested multiple times with different numbers of attributes.
Quality parameters  are computed for each case and plotted against the number of features in \autoref{fig:NaiveBayes_features}.

\begin{figure}[H]
  \centering
  \includegraphics[width=0.9\textwidth]{build/Naive_Bayes/Naive_Bayes_features.pdf}
  \caption{Quality parameters of the Naive Bayes classifier for a varying number of features.}
  \label{fig:NaiveBayes_features}
\end{figure}

\begin{figure}[H]
  \centering
  \includegraphics[width=0.6\textwidth]{build/Naive_Bayes/Naive_Bayes_confusion.pdf}
  \caption{TODO.}
  \label{fig:NaiveBayes_confusion}
\end{figure}

\subsubsection{Random Forest}

\begin{figure}[H]
  \centering
  \includegraphics[width=0.9\textwidth]{build/Random_Forest/Random_Forest_features.pdf}
  \caption{Quality parameters of the Random Forest for a varying number of features.}
  \label{fig:RandomForest_features}
\end{figure}

\begin{figure}[H]
  \centering
  \includegraphics[width=0.6\textwidth]{build/Random_Forest/Random_Forest_confusion.pdf}
  \caption{TODO.}
  \label{fig:RandomForest_confusion}
\end{figure}

\subsubsection{Neural Network}

\begin{figure}[H]
  \centering
  \includegraphics[width=0.9\textwidth]{build/Neural_Network/Neural_Network_features.pdf}
  \caption{Quality parameters of the Neural Network for a varying number of features.}
  \label{fig:NeuralNetwork_features}
\end{figure}

\begin{figure}[H]
  \centering
  \includegraphics[width=0.6\textwidth]{build/Neural_Network/Neural_Network_confusion.pdf}
  \caption{TODO.}
  \label{fig:NeuralNetwork_confusion}
\end{figure}

\subsection{Classification of test data}

\begin{figure}[H]
  \centering
  \includegraphics[width=0.7\textwidth]{build/roc_curves.pdf}
  \caption{TODO.}
  \label{fig:roc_curves}
\end{figure}
