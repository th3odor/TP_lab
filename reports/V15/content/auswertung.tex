\section{Analysis}
\label{sec:Analysis}

As a first step all of the aquired \textit{.h5} data is converted into \textit{.txt} files with
the help of the provided \textit{execute.py} script.

\subsection{Depletion voltage}

The first part of the analysis is the measurement of the current-voltage characteristic of the
silicon strip sensor. In \autoref{fig:leakage}, the measured leakuge current is displayed with
the corresponding applied bias voltage. A flattening of the curve at $\qty{60}{\volt}$ is cleary
visible, this corresponds exactly to the depletion voltage $U_{\text{dep}}=\qty{60}{\volt}$ of
the chip stated by the manufacturer. For the following measurements, the bias voltage is
set to $\qty{80}{\volt}$ to ensure the chip is always fully depleted.

\begin{figure}[H]
  \centering
  \includegraphics[width=0.6\textwidth]{build/leakage.pdf}
  \caption{Plot of the measured current-voltage characteristic.}
  \label{fig:leakage}
\end{figure}

\subsection{Pedestal run}

Next, the pedestal run is evaluated to determine the noise of the strips. The pedestal
for each strip is calculated by taking the mean value of the ADC counts of all events for each individual strip,
as given by \eqref{eq:pedestal}. The common mode shift for each event is then aquired by substracting
the pedestal from the ADC counts of each strip and again taking the mean value according to
\eqref{eq:common_mode}. At last, the noise is determined using the previous results by
plugging them into \eqref{eq:noise}. A bar diagram of the pedestals and noise is shown in
\autoref{fig:pedestal_run}.

\begin{figure}[H]
  \centering
    \begin{subfigure}{0.45\textwidth}
      \includegraphics[width=\textwidth]{build/pedestal.pdf}
      \caption{}
      \label{fig:pedestals}
    \end{subfigure}
    \begin{subfigure}{0.45\textwidth}
      \includegraphics[width=\textwidth]{build/noise.pdf}
      \caption{}
      \label{fig:noise}
    \end{subfigure} 
  \caption{Bar diagrams of the pedestals (a) and noise (b) of the 128 individual strips.}
  \label{fig:pedestal_run}
\end{figure}

The diagrams show a visible increase of the pedestals and noise towards the edges of the chip,
this might have something to with the structure of the chip or the way the signal is read out.
A histogram of the common mode shift is displayed in \autoref{fig:mode_shift}.

\begin{figure}[H]
  \centering
  \includegraphics[width=0.6\textwidth]{build/common_mode.pdf}
  \caption{Commong mode shift measured during the pedestal run.}
  \label{fig:mode_shift}
\end{figure}

As expected, the common mode shift also referred to as \textit{common noise} follows a gaussian
distribution around $0$.

\subsection{Calibration measurements}

Next, the data aquired from the calibration run is analysed. The results from the delay measurement
are shown in \autoref{fig:delay}.

\begin{figure}[H]
  \centering
  \includegraphics[width=0.6\textwidth]{build/delay.pdf}
  \caption{Results from the delay measurement.}
  \label{fig:delay}
\end{figure}

\autoref{fig:delay} shows the ADC counts depending on the delay of the laser. The highest
counts were archieved at $\qty{64}{\nano\second}$, which is why this value was set as the
laser delay for the rest of the experiment.
After this, a calibration curve is measured for five different channels. This curve gives
the relation between the injected charge and the ADC counts. The five curves are given
in \autoref{fig:calib}. Additionally for channel $60$, a curve at $\qty{0}{\volt}$ is
recorded, it can be seen in \autoref{fig:calib_0V}, where it is compared to the regular
curve of channel $60$.

\begin{figure}[H]
  \centering
    \begin{subfigure}{0.45\textwidth}
      \includegraphics[width=\textwidth]{build/calib.pdf}
      \caption{}
      \label{fig:calib}
    \end{subfigure}
    \begin{subfigure}{0.45\textwidth}
      \includegraphics[width=\textwidth]{build/calib_0V.pdf}
      \caption{}
      \label{fig:calib_0V}
    \end{subfigure} 
  \caption{Calibration curve of five channels (a) and detailed view of channel $60$ at $\qty{0}{\volt}$ (b).}
  \label{fig:calib_run}
\end{figure}

As seen in \autoref{fig:calib}, all channels show almost exactly the same calibration curve.
For some unknown reason, channel $20$ meausured significantly less ADC counts.
For the next step, the mean value of the five channels is taken. The calibration curve at
$\qty{0}{\volt}$ is very similar to the one measured above the depletion voltage. This
leads to the conclusion that the relation between the ADC counts and the injected charge does not
depend on the applied bias voltage.
To further quantify this dependecy of the counts and the injected charge, a 4-th degree
polynomial is fit to the mean counts of the five channels. The polynomial is of the from
\begin{equation}
  Q(\text{ADC}) = a\cdot\text{ADC}^4 + b\cdot\text{ADC}^3 + c\cdot\text{ADC}^2 + d\cdot\text{ADC} + e \, .
  \label{eq:polynomial}
\end{equation}
The fit is performed using the \textit{curvefit} method from the \textit{scipy} \cite{scipy}
package. The result is shown in \autoref{fig:calib_fit}, the fit range is restricted to the range
below $250$ ADC counts to get a good result.

\begin{figure}[H]
  \centering
  \includegraphics[width=0.6\textwidth]{build/calib_fit.pdf}
  \caption{Fit of 4-th degree polynomial to the mean ADC counts of the calibration run.}
  \label{fig:calib_fit}
\end{figure}

The fit yields the following coefficients for \eqref{eq:polynomial}
\begin{align*}
  a &= \qty{0.000111+-0.000006}{\elementarycharge} \\
  b &= \qty{-0.0439+-0.0032}{\elementarycharge} \\
  c &= \qty{6.2+-0.6}{\elementarycharge} \\
  d &= \qty{30+-40}{\elementarycharge} \\
  e &= \qty{2600+-700}{\elementarycharge} \, .
\end{align*}
These coefficients together with \eqref{eq:polynomial} now allow any ADC counts
to be converted into an electric charge.

\subsection{Measuring the strip sensor by using the laser}

The physical structure of the sensor and its strips is determined by using the laser.
Before the measurement can be started, the optimal delay between laser signal and chip readout
is determined by plotting the data from the \textit{Laser Sync} run. This is shown in
\autoref{fig:laser_delay}, where it is cleary visible that the maxiaml ADC counts are archieved at a
delay of $\qty{100}{\nano\second}$.

\begin{figure}[H]
  \centering
  \includegraphics[width=0.6\textwidth]{build/laser_delay.pdf}
  \caption{Results of the \textit{Laser Sync} run.}
  \label{fig:laser_delay}
\end{figure}

After moving the laser across the $\qty{35}{\micro\meter}$ intervall, a heatmap is produced
showing the signal of the relevant channels that were hit by the laser. 
\autoref{fig:laser_scan} shows the affected channels 82-85 and their signal strength depending
on the laser position.

\begin{figure}[H]
  \centering
  \includegraphics[width=0.7\textwidth]{build/laser_scan.pdf}
  \caption{Heatmap of the signal strength of the affected channels depending on the laser position.}
  \label{fig:laser_scan}
\end{figure}

In the following, the individual signal for channel 83 is analysed in \autoref{fig:channel_83}.
The position of the beginning, peak and end of the signal is also shown.

\begin{figure}[H]
  \centering
  \includegraphics[width=0.6\textwidth]{build/channel_83.pdf}
  \caption{Signal strength of channel 83 depending on the laser position.}
  \label{fig:channel_83}
\end{figure}

The width of the strip is simply the distance between the two peaks, as this is the position where the laser
is on the channel but reflected by the metal of the strip. The extension of the laser can be estimated by
the distance between the start of a peak and its maximum. All relevant positions are marked in
\autoref{fig:channel_83} and the resulting values are
\begin{align*}
  \text{width of strips} &= \qty{290}{\micro\meter} - \qty{170}{\micro\meter} = \qty{120}{\micro\meter} \\
  \text{laser extension} &= \qty{170}{\micro\meter} - \qty{120}{\micro\meter} = \qty{50}{\micro\meter} \, .
\end{align*}
The distance of two strips can be determined by comparing the position of the maxima of two different channels
in \autoref{fig:laser_scan}
\begin{align*}
  \text{distance} &= \qty{40}{\micro\meter} \, .
\end{align*}
The pitch is then calculated as the sum of the distance between the strips and their width
\begin{align*}
  \text{pitch} &= \qty{120}{\micro\meter} + \qty{40}{\micro\meter} = \qty{160}{\micro\meter} \, .
\end{align*}

\subsection{Determination of the charge collection efficiency}

The charge collection efficiency of the laser is determined by measuring the relation between the
collected charge (ADC counts) and the applied bias voltage.

\subsubsection{Using the laser}

When using the laser, at first the channel on which the laser is focused during the measurement is to
be determined. By plotting a heatmap of the ADC counts for each channel, \autoref{fig:CCEL_channel}
cleary shows that the laser was focused on channel $73$.

\begin{figure}[H]
  \centering
  \includegraphics[width=0.7\textwidth]{build/CCEL_channel.pdf}
  \caption{Heatmap showing the ADC counts of each channel and bias voltage.}
  \label{fig:CCEL_channel}
\end{figure}

Thus in the following, only the data from channel $73$ is considered. In order to get the
charge collection efficiency, the measured ADC counts
have to be normalized in regard to the maximum ADC counts of the plateu.
In theory, the plateu should show constant counts after the depletion voltage $U_{\text{dep}}$ is reached,
however the acutal data shows a slight increase in this area. This is why the counts
are normalized in regard to the first value of the plateu, so that a proper fit of 
\eqref{eq:charge_col_eff} can be performed. The normalized counts in dependence of the
bias voltage are shown in \autoref{fig:CCEL}.

\begin{figure}[H]
  \centering
  \includegraphics[width=0.6\textwidth]{build/CCEL.pdf}
  \caption{Charge collection efficiency of channel $73$.}
  \label{fig:CCEL}
\end{figure}

The fit was performed in the range $\qty{0}{\volt} < U < \qty{80}{\volt}$.
The depletion voltage of $U_{\text{dep}} = \qty{80}{\volt}$ is a fixed paramter in 
\eqref{eq:charge_col_eff}, whereas the value of the penetration depth $a$ of the laser is determined
via the fit as
\begin{align*}
  a &= \qty{253+-6}{\micro\meter} \, .
\end{align*}

\subsubsection{Using the beta source}

The same curve is now measured by replacing the laser as a source with a $\beta^-$ source.
The ADC counts are now determined in clusters. To get the charge collection efficiency,
the entries in each cluster are summed, after this the mean value of all clusters is taken.
A normalization is performed analogously to the one of laser measurement.
The resulting curve is displayed in \autoref{fig:CCEQ}.

\begin{figure}[H]
  \centering
  \includegraphics[width=0.6\textwidth]{build/CCEQ.pdf}
  \caption{Charge collection efficiency determined via the $\beta^-$ source.}
  \label{fig:CCEQ}
\end{figure}

\subsection{Large source scan}

Finally, the large source scan containing roughly one million events is analysed.
\autoref{fig:RS_histograms} shows how the number of cluster per event is distributed
as well as how many channels exist per cluster.

\begin{figure}[H]
  \centering
    \begin{subfigure}{0.45\textwidth}
      \includegraphics[width=\textwidth]{build/num_clusters.pdf}
      \caption{}
    \end{subfigure}
    \begin{subfigure}{0.45\textwidth}
      \includegraphics[width=\textwidth]{build/num_channels.pdf}
      \caption{}
    \end{subfigure}
  \caption{Number of clusters per event (a) and number of channels per cluster (b).}
  \label{fig:RS_histograms}
\end{figure}

Next, a hitmap is produced for all channels. This is displayed in
\autoref{fig:hitmap} and shows how many events/hits were registered by each
individual channel. An events mostly leads to the formation of one cluster,
and each cluster usually contains the signal of one to three channels.

\begin{figure}[H]
  \centering
  \includegraphics[width=0.5\textwidth]{build/hitmap.pdf}
  \caption{Hitmap displaying the events of each channel.}
  \label{fig:hitmap}
\end{figure}

Most of the hits are registered in the middle of the chip, while the
edges cleary show a lower amount of entries.
Lastly, the distribution of the measured ADC counts as well as the energy
spectrum is computed. To aquire the energy spectrum from the ADC counts, the
counts are converted into an electric pulse with the conversion formula \eqref{eq:polynomial} determined
during the calibration run. The pulse is then converted into an energy by
using the fact that the energy needed to generate an electron-hole pair is approximatly
$\qty{3.6}{\electronvolt}$ \cite{V15}. \autoref{fig:RS_distributions} shows
both distributions.

\begin{figure}[H]
  \centering
    \begin{subfigure}{0.45\textwidth}
      \includegraphics[width=\textwidth]{build/ADC_counts.pdf}
      \caption{}
    \end{subfigure}
    \begin{subfigure}{0.45\textwidth}
      \includegraphics[width=\textwidth]{build/energy.pdf}
      \caption{}
    \end{subfigure}
  \caption{Distribution of the ADC counts (a) and the energy (b).}
  \label{fig:RS_distributions}
\end{figure}

The distribution of the energ spectrum allows to determine the most probable energy
and mean energy of the $\beta^-$ source as
\begin{align*}
  E_{\mathrm{MPV}} &= \qty{97}{\kilo\electronvolt} \\
  E_{\mathrm{Mean}} &= \qty{163}{\kilo\electronvolt} \, .
\end{align*}