\section{Analysis}
\label{sec:Analysis}

As a first step all of the aquired \textit{.h5} data is converted into \textit{.txt} files with
the help of the provided \textit{execute.py} script.

\subsection{Depletion voltage}

The first part of the analysis is the measurement of the current-voltage characteristic of the
silicon strip sensor. In \autoref{fig:leakage}, the measured leakuge current is displayed with
the corresponding applied bias voltage. A flattening of the curve at $\qty{60}{\volt}$ is cleary
visible, this corresponds exactly to the depletion voltage $U_{\text{dep}}=\qty{60}{\volt}$ of
the chip stated by the manufacturer. For the following measurements, the bias voltage is
set to $\qty{80}{\volt}$ to ensure the chip is always fully depleted.

\begin{figure}[H]
  \centering
  \includegraphics[width=0.6\textwidth]{build/leakage.pdf}
  \caption{Plot of the measured current-voltage characteristic.}
  \label{fig:leakage}
\end{figure}

\subsection{Pedestal run}

Next, the pedestal run is evaluated to determine the noise of the strips. The pedestal
for each strip is calculated by taking the mean value of the ADC counts of all events for each individual strip,
as given by \eqref{eq:pedestal}. The common mode shift for each event is then aquired by substracting
the pedestal from the ADC counts of each strip and again taking the mean value according to
\eqref{eq:common_mode}. At last, the noise is determined using the previous results by
plugging them into \eqref{eq:noise}. A bar diagram of the pedestals and noise is shown in
\autoref{fig:pedestal_run}.

\begin{figure}[H]
  \centering
    \begin{subfigure}{0.45\textwidth}
      \includegraphics[width=\textwidth]{build/pedestal.pdf}
      \label{fig:pedestals}
    \end{subfigure}
    \begin{subfigure}{0.45\textwidth}
      \includegraphics[width=\textwidth]{build/noise.pdf}
      \label{fig:noise}
    \end{subfigure} 
  \caption{Bar diagrams of the pedestals and noise of the 128 individual strips.}
  \label{fig:pedestal_run}
\end{figure}

The diagrams show a visible increase of the pedestals and noise towards the edges of the chip,
this might have something to with the structure of the chip or the way the signal is read out.
A histogram of the common mode shift is displayed in \autoref{fig:mode_shift}.

\begin{figure}[H]
  \centering
  \includegraphics[width=0.6\textwidth]{build/common_mode.pdf}
  \caption{Commong mode shift measured during the pedestal run.}
  \label{fig:mode_shift}
\end{figure}

As expected, the common mode shift also referred to as \textit{common noise} follows a gaussian
distribution around $0$.

\subsection{Calibration measurements}



\begin{figure}[H]
  \centering
  \includegraphics[height=5cm]{build/delay.pdf}
  \caption{TODO.}
  \label{fig:delay}
\end{figure}

\begin{figure}[H]
  \centering
  \includegraphics[height=5cm]{build/calib.pdf}
  \caption{TODO.}
  \label{fig:calib}
\end{figure}

\begin{figure}[H]
  \centering
  \includegraphics[height=5cm]{build/calib_0V.pdf}
  \caption{TODO.}
  \label{fig:calib_0V}
\end{figure}

\begin{figure}[H]
  \centering
  \includegraphics[height=5cm]{build/calib_fit.pdf}
  \caption{TODO.}
  \label{fig:calib_fit}
\end{figure}

\subsection{Measuring the strip sensor by using the laser}

\begin{figure}[H]
  \centering
  \includegraphics[height=5cm]{build/laser_delay.pdf}
  \caption{TODO.}
  \label{fig:laser_delay}
\end{figure}

\begin{figure}[H]
  \centering
  \includegraphics[height=5cm]{build/laser_scan.pdf}
  \caption{TODO.}
  \label{fig:laser_scan}
\end{figure}

\begin{figure}[H]
  \centering
  \includegraphics[height=5cm]{build/channel_83.pdf}
  \caption{TODO.}
  \label{fig:channel_83}
\end{figure}

\subsection{Determination of the charge collection efficiency}

\subsubsection{Using the laser}

\begin{figure}[H]
  \centering
  \includegraphics[height=5cm]{build/CCEL_channel.pdf}
  \caption{TODO.}
  \label{fig:CCEL_channel}
\end{figure}

\begin{figure}[H]
  \centering
  \includegraphics[height=5cm]{build/CCEL.pdf}
  \caption{TODO.}
  \label{fig:CCEL}
\end{figure}

\subsubsection{Using the beta source}

\begin{figure}[H]
  \centering
  \includegraphics[height=5cm]{build/CCEQ.pdf}
  \caption{TODO.}
  \label{fig:CCEQ}
\end{figure}

\subsection{Large source scan}

\begin{figure}[H]
  \centering
    \begin{subfigure}{0.5\textwidth}
      \includegraphics[width=\textwidth]{build/num_clusters.pdf}
    \end{subfigure}
    \begin{subfigure}{0.5\textwidth}
      \includegraphics[width=\textwidth]{build/num_channels.pdf}
    \end{subfigure}
  \caption{TODO.}
  \label{fig:RS_histograms}
\end{figure}

\begin{figure}[H]
  \centering
  \includegraphics[height=5cm]{build/hitmap.pdf}
  \caption{TODO.}
  \label{fig:hitmap}
\end{figure}

\begin{figure}[H]
  \centering
    \begin{subfigure}{0.5\textwidth}
      \includegraphics[width=\textwidth]{build/ADC_counts.pdf}
    \end{subfigure}
    \begin{subfigure}{0.5\textwidth}
      \includegraphics[width=\textwidth]{build/energy.pdf}
    \end{subfigure}
  \caption{TODO.}
  \label{fig:RS_distributions}
\end{figure}