\section{Discussion}
\label{sec:Discussion}

\subsection{Depletion voltage}

In the first part, the depletion voltage is determined to be $\qty{60}{\volt}$.
Later, the depletion voltage is again measured by using the charge collection efficiency.
In this case, a value of $\qty{80}{\volt}$ is obtained. The manufacturer \cite{V15} states a depletion
voltage in the range of $\qty{60}{}$ to $\qty{80}{\volt}$, so both measurements are
acceptable. Both values are only estimations, as they are obtained by roughly determining the
point where a curve flattens into a plateau. Therefore it is expected that they do not fully agree.

\subsection{Measuring the strip sensors}

The pitch of the strip sensors has been calculated from their measured width and distance to be
$\qty{160}{\micro\meter}$. This is the exact value given by the manufacturer \cite{V15}, the high
precision could be achieved through the $\qty{10}{\micro\meter}$ resolution chosen for the scan.
The laser extension was measured at $\qty{50}{\micro\meter}$ and is thus significantly larger
than the stated \cite{V15} $\qty{20}{\micro\meter}$. This indicates that the laser was probably not
fully focused during the measurement. A reason for this could be the fact that it was difficult to find an
absolout maxima when adjusting the horizontal micrometer scew.

The charge collection efficiency curve shows a behaviour close to the theoretical one, with the
exception that the plateau slighty increases for voltages above the depletion voltage. This might be
a consequence of the noise, that increases with the bias voltage and is not fully removed from the data.

The value for the penetration depth $ a= \qty{253+-30}{\micro\meter}$ seems consistent with the sensor
thickness of $D = \qty{300}{\micro\meter}$ and the given pentration depths of lasers with slightly different
wavelenghts in \cite{V15}.

\subsection{Energy spectrum}

During the large source scan, the calculated energy spectrum resembles the expected convolution of a
Landau and a Gauss distribution \cite{V15}, where the most probable energy is lower than the mean energy.
Using the modified Bethe-Bloch equation, it can be calculated that the average energy disposition 
of an ionising electron with the maximum energy of the primary $^{90}\text{Sr}$ decay in pure silicon is
$\qty{3.88}{\mega\electronvolt\per\centi\meter}$.
Multiplying this value by the sensor thickness $D$, the theoretically expected mean energy loss becomes
$\bar{E}_{\text{theo}} = \qty{116.4}{\kilo\electronvolt}$. This can be compared to the measured
mean energy loss
\begin{align*}
    \bar{E}_{\text{theo}} &= \qty{116.4}{\kilo\electronvolt} \\
    \bar{E}_{\text{exp}} &= \qty{146.5+-0.6}{\kilo\electronvolt}\, .
\end{align*}
A deviation of $\qty{20.5}{\percent}$ is observed, which is quite large. A possible reason might be an underestimation
of the noise in the sensor. Furthermore the modified Bethe-Bloch equation is only an approximation with limited
precision.