%\section{Discussion}
%\label{sec:Discussion}
%%Overall the results obtained during the analysis seem to align with the expected results.
%%The applied analysis was able to quantify the theoretical predicted symmetry. 
%Looking at the requirements placed \texttt{prob} variables of the data were left untouched as lowering the background in \autoref{fig:fits} would not allow for a good fit on the background distribution. \\
%
%The distribution obtained from the invariant mass resemble the expected one, meaning an exponential function for the background and an gaussian bell curve for the signal candidates. These functions were then fitted onto the data. From these fits the global asymmetry factor $A_\mathrm{glo} = \num{0.04469\pm0.00815}$ obtained with a significance of $S = 5.48$. This result fulfills the 5\,$\sigma$ requirement placed on claiming a discovery.\\
%
%The Dalitz plots of both the simulation sample as well as the data sample also meet the expectation. For the simulation sample the Dalitz plot looks mostly homogeneously, whereas the Dalitz plot obtained from the data sample shows the expected symmetry as well as some lines from certain resonances determined int the section above. These resonances become more visible in the alternative Dalitz plot shown in \autoref{fig:dalitzalt}. The resonances were then removed from the data allowing for a more clean data sample.\\
%
%The last part of the analysis was the determination of the local asymmetry factor $A_\mathrm{loc} = \num{0.06286\pm0.01017}$ with a significance of $S = 6.18$. This significance is even higher than the one calculated for the global asymmetry. Even the asymmetry factor $A_\mathrm{loc}$ of the candidates in the determined high significance area was larger. This may be a result of the removal of the resonances, which do not contribute to the asymmetry.\\
%
%All in all it can be said that the analysis was able to produce good results with both the local and global asymmetry factor calculation resulted in a significance $>5$.


\section{Discussion} %grammerly corrected
\label{sec:Discussion}
%Overall the results obtained during the analysis seem to align with the expected results.
%The applied analysis was able to quantify the theoretical predicted symmetry. 
Looking at the requirements placed \texttt{prob} variables of the data were 
left untouched as lowering the background in \autoref{fig:fits} would not 
allow for a good fit on the background distribution. \\

The distribution obtained from the invariant mass resembles the expected one, 
meaning an exponential function for the background and a Gaussian bell curve 
for the signal candidates. These functions were then fitted onto the data. 
From these fits the global asymmetry factor $A_\mathrm{glo} = \num{0.04469\pm0.00815}$ 
obtained with a significance of $S = 5.48$. This result fulfills the 5\,$\sigma$ 
requirement placed on claiming a discovery.\\

The Dalitz plots of both the simulation sample as well as the data sample 
also meet the expectation. For the simulation sample, the Dalitz plot looks 
mostly homogeneously, whereas the Dalitz plot obtained from the data sample 
shows the expected symmetry as well as some lines from certain resonances 
determined in the section above. These resonances become more visible in the 
alternative Dalitz plot shown in \autoref{fig:dalitzalt}. 
The resonances were then removed from the data allowing for a cleaner data sample.\\

The last part of the analysis was the determination of the local asymmetry factor 
$A_\mathrm{loc} = \num{0.06286\pm0.01017}$ with a significance of $S = 6.18$. 
This significance is even higher than the one calculated for the global asymmetry. 
Even the asymmetry factor $A_\mathrm{loc}$ of the candidates in the determined high 
significance area was larger. This may result from the removal of the resonances, 
which do not contribute to the asymmetry.\\

All in all, it can be said that the analysis was able to produce good results 
with both the local and global asymmetry factor calculation resulting in a significance $>5$.


