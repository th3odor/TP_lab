\section{Objective}

During the Big Bang, equal amounts of matter and antimatter should have been created. However, our universe almost exclusively contains regular matter.
This so-called matter-antimatter problem remains unsolved until today. It can partially be explained by the violation of
the matter-antimatter (CP) symmetry in weak decays. This analysis uses data from weak $B$ meson decays recorded at the LHCb experiment to quantify
the CP asymmetry.

\section{Theory}
\label{sec:Theorie}

To understand how weak $B$ meson decays at LHCb can be used to study CP violation, it is necessary to explain the origin of CP violation in the standard model (SM)
of particle physics first. The following section gives a short overview about how CP violation was discovered and the way it is parameterized in the SM.

\subsection{The CKM matrix and CP violation}

In 1973, Kobayashi and Maskawa \cite{CKM_paper} completed the quark sector of the SM by introducing a third generation of quarks, the top ($t$) and bottom ($b$) quark.
Thus in total, three up-type quarks ($u$, $c$, $t$) and three down-type quarks ($d$, $s$, $b$) exist in the SM. During weak processes, the quarks couple to the
$W$ and $Z$ bosons in their weak eigenbasis. The weak eigenstates result from mixing the physical eigenstates of the quarks, this mixing is described
by the CKM matrix:
\begin{align*}
    \begin{pmatrix}
        d' \\
        s' \\
        b' \\
    \end{pmatrix}
    &=
    \underbrace{
    \begin{pmatrix}
        V_{\text{ud}} & V_{\text{us}} & V_{\text{ub}} \\
        V_{\text{cd}} & V_{\text{cs}} & V_{\text{cb}} \\
        V_{\text{td}} & V_{\text{ts}} & V_{\text{tb}} \\
    \end{pmatrix}
    }_{V_{\text{CKM}}}
    \cdot
    \begin{pmatrix}
        d \\
        s \\
        b \\
    \end{pmatrix}   . 
\end{align*}
Another way of interpreting this matrix is that each element quantifies how likely it is for a transition from one specific quark type to another to occur.
The CKM matrix is a complex and unitary $3 \times 3$ matrix, which is why it has four free parameters. Three of these parameters are the previously
mentioned quark mixing angles, however the fourth parameter is a complex phase $\delta$. This phase describes the CP violation and is the main reason why the third
generation of quarks was introduced. The CP is a short form for the charge (C) and parity (P) symmetry, which describes the fact that an antiparticle (regular particle
with opposite charge and parity) is expected to behave exactly the same as its regular particle counterpart. In 1964, violation of this CP-symmetry in weak Kaon decays was
observed and could not be described at first. The violation could finally explained through the implementation of the previously mentioned CP violation phase in 1973.

The CKM matrix elements have been experimentally measured and show a clear hierachy in respect to the different generations. A possible way of parameterizing the matrix
is the so-called Wolfenstein parameterization (at $\lambda^3$ order):
\begin{align*}
    V_{\text{CKM}} & \simeq
    \begin{pmatrix}
        1-\frac{1}{2}\lambda^2 & \lambda & A\lambda^3(\rho-\symup{i}\eta) \\
        -\lambda & 1-\frac{1}{2}\lambda^2 & A\lambda^2 \\
        A\lambda^3(1-\rho-\symup{i}\eta) & -A\lambda^2 & 1 \\
    \end{pmatrix}.
\end{align*}
The complex CP violating phase is represented in the parameters $\rho$ and $\eta$ in this case. It is important to note that these parameters only appear in the matrix elements
$V_{\text{ub}}$ and $V_{\text{td}}$ which will become important later.

\subsection{CP violation in weak decays of B mesons}

This analysis aims to study CP violation in weak $B$-meson decays. More specifically, decays of the $B^+$ meson and its CP counterpart $B^-$ of the following form
are considered:
\begin{align*}
    B^+ &\rightarrow h^+ h^+ h^- \\
    B^- &\rightarrow h^- h^- h^+
\end{align*}
Without CP violation, one would expect to observe the same number of $B^+$ decays ($N^+$) as $B^-$ decays ($N^-$). However, during the weak decay of the
$B$ meson consisting of a $b$ and $u$ quark, quark transitions of the form $b \rightarrow u$ can occur. The corresponding CKM matrix element $V_{\text{ub}}$
is a complex number, as explained in the previous section. Consequently, the matrix element differs if a different CP variant of the decay is considered, due
to the complex conjugation involved. In practice, the observed counts $N^+$ and $N^-$ will be different and a CP asymmetry can be defined \cite{LHCbCPV} as:
\begin{equation}
    A_{\text{CP}} = \frac{N^+ - N^-}{N^- + N^+}.
    \label{eq:asymmetry}
\end{equation}
During the $B$ decay, intermediate resonances contaning $c$ quarks can occur due to e.g. quark transitions of the form $b \rightarrow c$. Examples for this are
different $D$-mesons or charmonium ($c \bar{c}$) resonances. These resonances have to be removed, because the CP violation is much smaller in this case. The reason for this is
the CKM matrix element $V_{\text{cb}}$ that does not contain the CP violating parameters $\rho$ and $\eta$ (at the given order).